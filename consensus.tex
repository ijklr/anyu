    The consensus is the most crucial part of decentralized peer-to-peer network and is
imperative for a permission-less blockchain. Due to physical hardware and human factors,
latency, downtime and malicious attack are prevalent among distributed systems. When
two nodes have inconsistent block records, a fork occurs. This can cause double-spending: if
node A contains a transaction which is not found on node B, then who is to make the
determination of whether or not the transaction was valid? To address this, a blockchain
requires every participant follow the exact same rules in order to keep their state stay
synchronized. Currently, common blockchain consensus mechanisms include Proof-of-Work,
Proof-of-Stake and Byzantine Fault Tolerance. 
Proof-of-work was first proposed by Hashcash as an anti-spamming technique, and
later saw greater adoption in Bitcoin[1] and Ethereum[2]. It is the classic and prevalent
consensus mechanism within the current blockchain landscape. The principle of Proof-ofWork
requires a hard-to-obtain, while easy-to-verify hash value to be located by a node
before it is accepted by its peers. This method is designed in order to prevent the 'Sybil
attack', wherein an attacker can produce millions of fake nodes to replace honest ones.
Specifically, a Proof-of-Work is obtained through repeated calculations of the block header
until a hash less than a certain difficulty is found, i.e., the “mining” process. All nodes
consider the blockchain containing the most recent Proof-of-Work to be the “real” chain.
Therefore, to control the blockchain, an attacker must possess more than 51% total
computing power of the entire network. This mechanism is also called permission-less
consensus, which allows nodes to join and exit anonymously at any time.
Although Proof-of-Work is one of the most secure permission-less consensus algorithms
to date, its drawbacks will likely hinder mainstream adaptation for high volume
applications that coincide with mainstream adoption. Currently, The Bitcoin network
generates a block every ten minutes, with a maximum block size of 1MB. This has
restricted Bitcoin’s transaction speed to no more than 7 transactions per second. Moreover,
because of Bitcoin’s inherent forking problem, a transaction can be considered secure only
after six confirmations, which takes a minimum of an hour. This performance is intolerable
for modern payment processing. In contrast, VisaNet is capable of processing 2000
transactions per second on average and may handle over 10000 transactions per second
during its peak. However, nearly transactions are validated in one-of-two secure facilities in
the U.S. The implications of raising the block-size have been debated within the Bitcoin
community for several years now. However, successfully doing so would require a
significantly large sum of the network to change their software, effectively creating a fork of
the Bitcoin blockchain, until the entire network resolves the leader. Two popular and
seemingly acceptable solutions are side-chains, and lightning networks, both of which
represent off-chain solutions and inevitably bring centralization and security problems.
Because Proof-of-Work blockchains currently consume a relatively large amount of
energy, there have also been proposals of consensus on the ownership of stakes[3], or, 'coins'.
In this schema, nodes express their acceptance of a transaction by locking up a portion of
their coins as security deposit. These nodes, also called stakeholders, serve as validators for
all incoming transactions. If a stakeholder commits fraud, its deposit is forfeited as
punishment. However, since developers control the initial distribution of coins, Proof-ofStake
is inherently more centralized than Proof-of-Work. Stakeholders tend to store more
tokens, reducing the supply of tokens in circulation. Moreover, current implementations of 
Proof-of-Stake have proven far less secure than Proof-of-Work because any stakeholder can
initiate a nothing-at-stake attack by signing multiple blockchain histories. The centralized
stakeholders also create single-point-of-failure. If stakeholders collude together or are taken
down by hackers or authority, the entire network goes down as well. Although Proof-ofStake
has been proposed as early as 2012 by Peercoin[4], it has not seen as wide of an
acceptance as the Proof-of-Work mechanism.
The consensus problem may also be generalized as the Byzantine general’s problem,
wherein nodes within a system can exhibit any abnormal behavior while the system must
continue to function normally. It has been proven that in order to tolerate f number of
Byzantine nodes, at least 3f+1 nodes must exist in the system. Within the peer-to-peer
network of the blockchain, nodes can be disconnected, shutdown or broadcast malicious
information. It is impossible to predict their behavior. In 1999, researchers at MIT for the
first time published the Practical Byzantine Fault Tolerance algorithm[5], which reduced the
complexity of Byzantine algorithm from exponential to polynomial. This algorithm chooses
one node as primary and a group of nodes as backup. The consensus is achieved after
multiple rounds of voting amongst nodes. This method was originally designed for internal
closed systems, such as flight control system of airplanes and data center clusters. It is not
suitable for decentralized, peer-to-peer applications such as cryptocurrencies. Better-known
examples of Byzantine fault tolerance based blockchains are Hyperledger, Tendermint[6]
and Stellar[7]. In these examples, traditional mining is abandoned and transaction speeds of
up to 100,000 tx/s are possible. Similarly, PBFT does not support nodes to join and exit the
network at will. And when the total number of nodes go beyond 20, the entire network
drastically slows down [8]. 
