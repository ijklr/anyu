%%% DOCUMENTCLASS 
%%%-------------------------------------------------------------------------------

\documentclass[
a4paper, % Stock and paper size.
11pt, % Type size.
% article,
% oneside, 
onecolumn, % Only one column of text on a page.
% openright, % Each chapter will start on a recto page.
% openleft, % Each chapter will start on a verso page.
openany, % A chapter may start on either a recto or verso page.
]{memoir}

%%% PACKAGES 
%%%------------------------------------------------------------------------------

\usepackage[utf8]{inputenc} % If utf8 encoding
% \usepackage[lantin1]{inputenc} % If not utf8 encoding, then this is probably the way to go
\usepackage[T1]{fontenc}    %
\usepackage[english]{babel} % English please
\usepackage[final]{microtype} % Less badboxes

% \usepackage{kpfonts} %Font

\usepackage{amsmath,amssymb,mathtools} % Math

% \usepackage{tikz} % Figures
\usepackage{graphicx} % Include figures

% code syntax 
\usepackage{listings}
\usepackage{framed}
\usepackage{minted}

%%% PAGE LAYOUT 
%%%------------------------------------------------------------------------------

\setlrmarginsandblock{0.15\paperwidth}{*}{1} % Left and right margin
\setulmarginsandblock{0.2\paperwidth}{*}{1}  % Upper and lower margin
\checkandfixthelayout

%%% SECTIONAL DIVISIONS
%%%------------------------------------------------------------------------------

\maxsecnumdepth{subsection} % Subsections (and higher) are numbered
\setsecnumdepth{subsection}

\makeatletter %
\makechapterstyle{standard}{
  \setlength{\beforechapskip}{0\baselineskip}
  \setlength{\midchapskip}{1\baselineskip}
  \setlength{\afterchapskip}{8\baselineskip}
  \renewcommand{\chapterheadstart}{\vspace*{\beforechapskip}}
  \renewcommand{\chapnamefont}{\centering\normalfont\Large}
  \renewcommand{\printchaptername}{\chapnamefont \@chapapp}
  \renewcommand{\chapternamenum}{\space}
  \renewcommand{\chapnumfont}{\normalfont\Large}
  \renewcommand{\printchapternum}{\chapnumfont \thechapter}
  \renewcommand{\afterchapternum}{\par\nobreak\vskip \midchapskip}
  \renewcommand{\printchapternonum}{\vspace*{\midchapskip}\vspace*{5mm}}
  \renewcommand{\chaptitlefont}{\centering\bfseries\LARGE}
  \renewcommand{\printchaptertitle}[1]{\chaptitlefont ##1}
  \renewcommand{\afterchaptertitle}{\par\nobreak\vskip \afterchapskip}
}
\makeatother

\chapterstyle{standard}

\setsecheadstyle{\normalfont\large\bfseries}
\setsubsecheadstyle{\normalfont\normalsize\bfseries}
\setparaheadstyle{\normalfont\normalsize\bfseries}
\setparaindent{0pt}\setafterparaskip{0pt}

%%% FLOATS AND CAPTIONS
%%%------------------------------------------------------------------------------

\makeatletter                  % You do not need to write [htpb] all the time
\renewcommand\fps@figure{htbp} %
\renewcommand\fps@table{htbp}  %
\makeatother                   %

\captiondelim{\space } % A space between caption name and text
\captionnamefont{\small\bfseries} % Font of the caption name
\captiontitlefont{\small\normalfont} % Font of the caption text

\changecaptionwidth          % Change the width of the caption
\captionwidth{1\textwidth} %

%%% ABSTRACT
%%%------------------------------------------------------------------------------

\renewcommand{\abstractnamefont}{\normalfont\small\bfseries} % Font of abstract title
\setlength{\absleftindent}{0.1\textwidth} % Width of abstract
\setlength{\absrightindent}{\absleftindent}

%%% HEADER AND FOOTER 
%%%------------------------------------------------------------------------------

\makepagestyle{standard} % Make standard pagestyle

\makeatletter                 % Define standard pagestyle
\makeevenfoot{standard}{}{}{} %
\makeoddfoot{standard}{}{}{}  %
\makeevenhead{standard}{\bfseries\thepage\normalfont\qquad\small\leftmark}{}{}
\makeoddhead{standard}{}{}{\small\rightmark\qquad\bfseries\thepage}
% \makeheadrule{standard}{\textwidth}{\normalrulethickness}
\makeatother                  %

\makeatletter
\makepsmarks{standard}{
\createmark{chapter}{both}{shownumber}{\@chapapp\ }{ \quad }
\createmark{section}{right}{shownumber}{}{ \quad }
\createplainmark{toc}{both}{\contentsname}
\createplainmark{lof}{both}{\listfigurename}
\createplainmark{lot}{both}{\listtablename}
\createplainmark{bib}{both}{\bibname}
\createplainmark{index}{both}{\indexname}
\createplainmark{glossary}{both}{\glossaryname}
}
\makeatother                               %

\makepagestyle{chap} % Make new chapter pagestyle

\makeatletter
\makeevenfoot{chap}{}{\small\bfseries\thepage}{} % Define new chapter pagestyle
\makeoddfoot{chap}{}{\small\bfseries\thepage}{}  %
\makeevenhead{chap}{}{}{}   %
\makeoddhead{chap}{}{}{}    %
% \makeheadrule{chap}{\textwidth}{\normalrulethickness}
\makeatother

\nouppercaseheads
\pagestyle{standard}               % Choosing pagestyle and chapter pagestyle
\aliaspagestyle{chapter}{chap} %

%%% NEW COMMANDS
%%%------------------------------------------------------------------------------

\newcommand{\p}{\partial} %Partial
% Or what ever you want

%%% TABLE OF CONTENTS
%%%------------------------------------------------------------------------------

\maxtocdepth{subsection} % Only parts, chapters and sections in the table of contents
\settocdepth{subsection}

\AtEndDocument{\addtocontents{toc}{\par}} % Add a \par to the end of the TOC

%%% INTERNAL HYPERLINKS
%%%-------------------------------------------------------------------------------

\usepackage{hyperref}   % Internal hyperlinks
\hypersetup{
pdfborder={0 0 0},      % No borders around internal hyperlinks
pdfauthor={I am the Author} % author
}
\usepackage{memhfixc}   %

%%% THE DOCUMENT
%%% Where all the important stuff is included!
%%%-------------------------------------------------------------------------------

\author{HDC Team @ hadeschain.org}
\title{Hades Chain: A highly scalable hybrid blockchain platform for decentralized financial \& data-intensive applications}

\usepackage{lipsum} % Just to put in some text

\begin{document}

\frontmatter

\maketitle

\begin{abstract}
We present an implementation of a new protocol(Hades Protocol) as a reliable, scalable platform for blockchain applications. First, we outline the limitations of current blockchain technologies and propose our solution: Hades Chain. Hades Chain is the first practical implemtation of the Hades Protocol, which addresses the biggest limitation of the most prominent cryptocurrencies such as bitcoin and ethereum - failure to achieve sufficient transaction throughput, or an application that leverages one. Hades Protocol is a hybrid consensus algorithm merging both PPoW(Proof of Proof of Work) and DPoS(Delegated Proof of Stake). In addition, we present a smart contract to show the unique mechanism of the Zero-Trust Collateralized Transaction System. 
\end{abstract}
\clearpage

\tableofcontents*

\clearpage

\chapter{Introduction}
Bitcoin revolutionized cryptocurrency, and Ethereum, the smart contract platform. We
are using our later mover advantage to learn from the mistakes and growing pains other
coins have experienced.
Both Bitcoin and Ethereum blockchains are based on a mathematical process called
Proof-of-Work[1]. Proof-of-work serves both as a consensus mechanism within a decentralized
network and as an incentive for nodes running it. While Proof-of-Work is a decentralized
way of distributing tokens, it is not without its shortcomings. The most significant
drawback is lack of scalability. As of June 2017, there are more than a quarter million
unconfirmed transactions within the Bitcoin network mempool. Executing a single Bitcoin
transaction can take between a few hours to a few days, with transaction fees averaging
several dollars and rising with the increasing demand of the block-size market. Because
Bitcoin’s parameters are hard-coded in its client, the Bitcoin network will split, or hardfork,
unless nodes upgrade their Bitcoin Client at once to, e.g., SegWit2X.
Ethereum faces similar structural problems and will eventually follow the same fate of
Bitcoin unless Ethereum adopts Proof-of-Stake. However, Proof-of-Stake is not without its
drawbacks, including increased incentives for centralization and the nothing-at-stake
problem. These shortcomings have become more evident in recent months, as Ethereum
smart contract execution fees (gas) recently rose to more than 5 US cents from less than one
cent in 2016. Other consensus models such as Delegated Proof-of-Stake are either
unsubstantiated or inevitably introduce one or more parties that must be trusted.
The Basic Attention Token (BAT) crowd sale raised over 35 million US dollars within
just two Ethereum blocks, or about 30 seconds. More than 90% percent of its coins went to
just a few people. For initial distribution, Cypherium will make every reasonable effort to
ensure the distributed ownership of tokens, consistent with our belief in decentralization.
High throughput, scalability, fast settlement and low fees -- while retaining
decentralization, security, trustlessness, pseudonymity, open membership and immutability
-- have become essential for the next generation of blockchain platforms to truly enable
enterprise level applications. A public blockchain must be able to upgrade itself without the
involvement of a centralized party, such as an official release of a newer client.
Security is another major issue of current public blockchains. Ethereum is the first
smart contract platform to see widespread adoption. The DAO attack in June 2016 raised
questions regarding whether blockchain history should be re-writeable to serve the
interests of a group of people. While the developers of the DAO were working on the solution
to fix its bugs, an attacker had already begun to take advantage of those bugs by
transferring over 3.6m ethers to his own account. In addition, Ethereum lacks a
decentralized, open and transparent governance mechanism that amends itself when it is
necessary to apply mitigation or upgrade.
In this white paper, we present a new blockchain, which addresses present
shortcomings of current public blockchain infrastructures. Key features of our design are
double-chain consensus, federated architecture and separate sandboxes for test and
production smart contracts.


\mainmatter

% Using typewriter font: \ttfamily inside \lstset

\chapter{Scalable Blockchain Systems}
\lipsum[1-10]
\chapter{Consensus Algorithm}
    The consensus is the most crucial part of decentralized peer-to-peer network and is
imperative for a permission-less blockchain. Due to physical hardware and human factors,
latency, downtime and malicious attack are prevalent among distributed systems. When
two nodes have inconsistent block records, a fork occurs. This can cause double-spending: if
node A contains a transaction which is not found on node B, then who is to make the
determination of whether or not the transaction was valid? To address this, a blockchain
requires every participant follow the exact same rules in order to keep their state stay
synchronized. Currently, common blockchain consensus mechanisms include Proof-of-Work,
Proof-of-Stake and Byzantine Fault Tolerance. 
Proof-of-work was first proposed by Hashcash as an anti-spamming technique, and
later saw greater adoption in Bitcoin[1] and Ethereum[2]. It is the classic and prevalent
consensus mechanism within the current blockchain landscape. The principle of Proof-ofWork
requires a hard-to-obtain, while easy-to-verify hash value to be located by a node
before it is accepted by its peers. This method is designed in order to prevent the 'Sybil
attack', wherein an attacker can produce millions of fake nodes to replace honest ones.
Specifically, a Proof-of-Work is obtained through repeated calculations of the block header
until a hash less than a certain difficulty is found, i.e., the “mining” process. All nodes
consider the blockchain containing the most recent Proof-of-Work to be the “real” chain.
Therefore, to control the blockchain, an attacker must possess more than 51% total
computing power of the entire network. This mechanism is also called permission-less
consensus, which allows nodes to join and exit anonymously at any time.
Although Proof-of-Work is one of the most secure permission-less consensus algorithms
to date, its drawbacks will likely hinder mainstream adaptation for high volume
applications that coincide with mainstream adoption. Currently, The Bitcoin network
generates a block every ten minutes, with a maximum block size of 1MB. This has
restricted Bitcoin’s transaction speed to no more than 7 transactions per second. Moreover,
because of Bitcoin’s inherent forking problem, a transaction can be considered secure only
after six confirmations, which takes a minimum of an hour. This performance is intolerable
for modern payment processing. In contrast, VisaNet is capable of processing 2000
transactions per second on average and may handle over 10000 transactions per second
during its peak. However, nearly transactions are validated in one-of-two secure facilities in
the U.S. The implications of raising the block-size have been debated within the Bitcoin
community for several years now. However, successfully doing so would require a
significantly large sum of the network to change their software, effectively creating a fork of
the Bitcoin blockchain, until the entire network resolves the leader. Two popular and
seemingly acceptable solutions are side-chains, and lightning networks, both of which
represent off-chain solutions and inevitably bring centralization and security problems.
Because Proof-of-Work blockchains currently consume a relatively large amount of
energy, there have also been proposals of consensus on the ownership of stakes[3], or, 'coins'.
In this schema, nodes express their acceptance of a transaction by locking up a portion of
their coins as security deposit. These nodes, also called stakeholders, serve as validators for
all incoming transactions. If a stakeholder commits fraud, its deposit is forfeited as
punishment. However, since developers control the initial distribution of coins, Proof-ofStake
is inherently more centralized than Proof-of-Work. Stakeholders tend to store more
tokens, reducing the supply of tokens in circulation. Moreover, current implementations of 
Proof-of-Stake have proven far less secure than Proof-of-Work because any stakeholder can
initiate a nothing-at-stake attack by signing multiple blockchain histories. The centralized
stakeholders also create single-point-of-failure. If stakeholders collude together or are taken
down by hackers or authority, the entire network goes down as well. Although Proof-ofStake
has been proposed as early as 2012 by Peercoin[4], it has not seen as wide of an
acceptance as the Proof-of-Work mechanism.
The consensus problem may also be generalized as the Byzantine general’s problem,
wherein nodes within a system can exhibit any abnormal behavior while the system must
continue to function normally. It has been proven that in order to tolerate f number of
Byzantine nodes, at least 3f+1 nodes must exist in the system. Within the peer-to-peer
network of the blockchain, nodes can be disconnected, shutdown or broadcast malicious
information. It is impossible to predict their behavior. In 1999, researchers at MIT for the
first time published the Practical Byzantine Fault Tolerance algorithm[5], which reduced the
complexity of Byzantine algorithm from exponential to polynomial. This algorithm chooses
one node as primary and a group of nodes as backup. The consensus is achieved after
multiple rounds of voting amongst nodes. This method was originally designed for internal
closed systems, such as flight control system of airplanes and data center clusters. It is not
suitable for decentralized, peer-to-peer applications such as cryptocurrencies. Better-known
examples of Byzantine fault tolerance based blockchains are Hyperledger, Tendermint[6]
and Stellar[7]. In these examples, traditional mining is abandoned and transaction speeds of
up to 100,000 tx/s are possible. Similarly, PBFT does not support nodes to join and exit the
network at will. And when the total number of nodes go beyond 20, the entire network
drastically slows down [8]. 

\lipsum[10 -25]
\chapter{Anonymization and Privacy}
\lipsum[20 -35]
\chapter{The Hades Virtual Machine}
\lipsum[30 -45]
\chapter{Zero-Trust Collateralized Transactions}
\lipsum[40 -55]
\chapter{Hades Protocol}
\lipsum[50 -65]
\chapter{User Level ICO}
\lipsum[60 -75]
\chapter{HDC Mining}
\lipsum[70 -85]
\chapter{HDC Lottery}
\lipsum[80 -95]
\chapter{Artificial Intelligence Smart Contracts}
\lipsum[90-105]
\chapter{Regulation}
\lipsum[100 - 105]
\chapter{Futuer-proof}
\lipsum[110 -125]
\appendix
\chapter{Calculations}
\begin{minted}{javascript}
{
"data": ["blob", "list", "blob"],
// lists have an array of object types as data
"links": [
{ "hash": "XLYkgq61DYaQ8NhkcqyU7rLcnSa7dSHQ16x",
"size": 189458 },
{ "hash": "XLHBNmRQ5sJJrdMPuu48pzeyTtRo39tNDR5",
"size": 19441 },
{ "hash": "XLWVQDqxo9Km9zLyquoC9gAP8CL1gWnHZ7z",
"size": 5286 }
// lists have no names in links
]
}
\end{minted}


\chapter{Language Reference: C++ API}
\begin{minted}{c++}
// Funtion that implements Dijkstra's single source shortest path algorithm
// for a graph represented using adjacency matrix representation
void dijkstra(int graph[V][V], int src)
{
     int dist[V];     // The output array.  dist[i] will hold the shortest
                      // distance from src to i
  
     bool sptSet[V]; // sptSet[i] will true if vertex i is included in shortest
                     // path tree or shortest distance from src to i is finalized
  
     // Initialize all distances as INFINITE and stpSet[] as false
     for (int i = 0; i < V; i++)
        dist[i] = INT_MAX, sptSet[i] = false;
  
     // Distance of source vertex from itself is always 0
     dist[src] = 0;
  
     // Find shortest path for all vertices
     for (int count = 0; count < V-1; count++)
     {
       // Pick the minimum distance vertex from the set of vertices not
       // yet processed. u is always equal to src in first iteration.
       int u = minDistance(dist, sptSet);
  
       // Mark the picked vertex as processed
       sptSet[u] = true;
  
       // Update dist value of the adjacent vertices of the picked vertex.
       for (int v = 0; v < V; v++)
  
         // Update dist[v] only if is not in sptSet, there is an edge from 
         // u to v, and total weight of path from src to  v through u is 
         // smaller than current value of dist[v]
         if (!sptSet[v] && graph[u][v] && dist[u] != INT_MAX 
                                       && dist[u]+graph[u][v] < dist[v])
            dist[v] = dist[u] + graph[u][v];
     }
  
     // print the constructed distance array
     printSolution(dist, V);
}
\end{minted}
\chapter{Language Reference: Java API}
\begin{minted}{java}
public static Graph calculateShortestPathFromSource(Graph graph, Node source) {
    source.setDistance(0);
 
    Set<Node> settledNodes = new HashSet<>();
    Set<Node> unsettledNodes = new HashSet<>();
 
    unsettledNodes.add(source);
 
    while (unsettledNodes.size() != 0) {
        Node currentNode = getLowestDistanceNode(unsettledNodes);
        unsettledNodes.remove(currentNode);
        for (Entry < Node, Integer> adjacencyPair: 
          currentNode.getAdjacentNodes().entrySet()) {
            Node adjacentNode = adjacencyPair.getKey();
            Integer edgeWeight = adjacencyPair.getValue();
            if (!settledNodes.contains(adjacentNode)) {
                calculateMinimumDistance(adjacentNode, edgeWeight, currentNode);
                unsettledNodes.add(adjacentNode);
            }
        }
        settledNodes.add(currentNode);
    }
    return graph;
}
\end{minted}
\backmatter

%%% BIBLIOGRAPHY
%%% -------------------------------------------------------------

% \bibliographystyle{utphysics}
% \bibliography{ref}

\end{document}