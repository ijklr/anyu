Bitcoin revolutionized cryptocurrency, and Ethereum, the smart contract platform. We
are using our later mover advantage to learn from the mistakes and growing pains other
coins have experienced.
Both Bitcoin and Ethereum blockchains are based on a mathematical process called
Proof-of-Work[1]. Proof-of-work serves both as a consensus mechanism within a decentralized
network and as an incentive for nodes running it. While Proof-of-Work is a decentralized
way of distributing tokens, it is not without its shortcomings. The most significant
drawback is lack of scalability. As of June 2017, there are more than a quarter million
unconfirmed transactions within the Bitcoin network mempool. Executing a single Bitcoin
transaction can take between a few hours to a few days, with transaction fees averaging
several dollars and rising with the increasing demand of the block-size market. Because
Bitcoin’s parameters are hard-coded in its client, the Bitcoin network will split, or hardfork,
unless nodes upgrade their Bitcoin Client at once to, e.g., SegWit2X.
Ethereum faces similar structural problems and will eventually follow the same fate of
Bitcoin unless Ethereum adopts Proof-of-Stake. However, Proof-of-Stake is not without its
drawbacks, including increased incentives for centralization and the nothing-at-stake
problem. These shortcomings have become more evident in recent months, as Ethereum
smart contract execution fees (gas) recently rose to more than 5 US cents from less than one
cent in 2016. Other consensus models such as Delegated Proof-of-Stake are either
unsubstantiated or inevitably introduce one or more parties that must be trusted.
The Basic Attention Token (BAT) crowd sale raised over 35 million US dollars within
just two Ethereum blocks, or about 30 seconds. More than 90% percent of its coins went to
just a few people. For initial distribution, Cypherium will make every reasonable effort to
ensure the distributed ownership of tokens, consistent with our belief in decentralization.
High throughput, scalability, fast settlement and low fees -- while retaining
decentralization, security, trustlessness, pseudonymity, open membership and immutability
-- have become essential for the next generation of blockchain platforms to truly enable
enterprise level applications. A public blockchain must be able to upgrade itself without the
involvement of a centralized party, such as an official release of a newer client.
Security is another major issue of current public blockchains. Ethereum is the first
smart contract platform to see widespread adoption. The DAO attack in June 2016 raised
questions regarding whether blockchain history should be re-writeable to serve the
interests of a group of people. While the developers of the DAO were working on the solution
to fix its bugs, an attacker had already begun to take advantage of those bugs by
transferring over 3.6m ethers to his own account. In addition, Ethereum lacks a
decentralized, open and transparent governance mechanism that amends itself when it is
necessary to apply mitigation or upgrade.
In this white paper, we present a new blockchain, which addresses present
shortcomings of current public blockchain infrastructures. Key features of our design are
double-chain consensus, federated architecture and separate sandboxes for test and
production smart contracts.
